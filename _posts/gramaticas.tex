\subsection{Gramaticas Regulares}\label{gramaticas-regulares}

Una gramática regular derecha es aquella cuyas reglas de producción P
son de la siguiente forma:

\begin{itemize}
\itemsep1pt\parskip0pt\parsep0pt
\item
  A → a, donde A es un símbolo no-terminal en N y a uno terminal en
  \(\Sigma\)\\
\item
  A → aB, donde A y B pertenecen a N y a pertenece a \(\Sigma\)\\
\item
  A → \(\epsilon\), donde A pertenece a N.
\end{itemize}

Análogamente, en una gramática regular izquierda, las reglas son de la
siguiente forma:

\begin{itemize}
\itemsep1pt\parskip0pt\parsep0pt
\item
  A → a, donde A es un símbolo no-terminal en N y a uno terminal en
  \(\Sigma\)
\item
  A → Ba, donde A y B pertenecen a N y a pertenece a \(\Sigma\)\\
\item
  A → \(\epsilon\), donde A pertenece a N.
\end{itemize}

Nota : Si mezclamos reglas de una regular derecha y una regular
izquierda lo que obtenemos\\sigue siendo una gramatica lineal pero no
necesariamente una gramatica regular.

Por ejemplo la gramatoca G con N = \{S, A\}, \(\Sigma\) = \{a, b\}, P
con simbolo inicial S y reglas:

\begin{itemize}
\itemsep1pt\parskip0pt\parsep0pt
\item
  S → aA\\
\item
  A → Sb\\
\item
  S → \(\epsion\)\\genera \(a^ib^i : i \geq 0\), que es un lenguaje
  lineal no regular.
\end{itemize}
